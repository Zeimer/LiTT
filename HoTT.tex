\documentclass{beamer}
\usepackage[utf8]{inputenc}
\usepackage{polski}
\usepackage[polish]{babel}
\usetheme{Darmstadt}

\newcommand{\impl}{\rightarrow}
\renewcommand{\implies}{\rightarrow}

\title{Homotopiczna teoria typów}

\author{Zeimer}
\date{13 stycznia 2019}

\begin{document}

\frame{\titlepage}

\frame{\tableofcontents}

\section{O co chodzi}

\begin{frame}{Odległe związki}
\begin{itemize}
	\item Homotopiczna teoria typów to połączenie teorii homotopii i teorii typów.
	\item Teorii typów nie trzeba nikomu przedstawiać.
	\item Teoria homotopii... TODO.
\end{itemize}
\end{frame}

\begin{frame}{Po co HoTT? 1}
\begin{itemize}
	\item HoTT jest też pomysłem na nowe podstawy matematyki (foundations of mathematics).
	\item Jest teorią typów, więc świetnie nadaje się do formalizacji za pomocą komputerów, czego nie można powiedzieć o teorii zbiorów.
	\item Jest syntetyczną teorią homotopii, co pozwala łatwo formalizować całkiem ezoteryczną matematykę.
	\item Jest konstruktywna, co powinno podobać się konstruktywistom. Co więcej, dzięki wyższym typom induktywnym pozwala na konstruktywne rozwiązanie problemów, które dotąd wymagały podejścia klasycznego.
\end{itemize}
\end{frame}

\begin{frame}{Po co HoTT? 2}
\begin{itemize}
	\item Dopuszcza niektóre klasyczne aksjomaty, co powinno przypaść do gusty klasykom.
	\item Jednocześnie podaje dobre filozoficzne podstawy (i formalne doowdy), by niektóre inne klasyczne aksjomaty odrzucić.
	\item Z trzeciej strony, pozwala wyrazić w teorii typów niektóre klasyczne aksjomaty, których dotychczas wyrazić nie było można (np. aksjomat wyboru).
\end{itemize}
\begin{theorem}
$\textstyle
	\prod (A : \mathcal{U}) (B : A \to \mathcal{U}) (R : \Pi x : A, B\ x \to \mathcal{U}),$ \\
		$(\prod x : A, \sum y : B\ x, R\ x\ y) \to$ \\
			$\sum f : (\Pi x : A, B x), \Pi x : A, R\ x\ (f\ x)$
\end{theorem}
\end{frame}


\begin{frame}{Pomysły}
\begin{itemize}
	\item O co chodzi z teorią homotopii i topologią algebraiczną? Grupa podstawowa okręgu.
	\item Hierarchia n-typów: unit, funkcje sortujące, pusty, zbiory, grupoidy, okrąg, uniwersum.
	
\end{itemize}
\end{frame}

\begin{frame}{wut}
\begin{itemize}
	\item
\end{itemize}
\end{frame}

\end{document}