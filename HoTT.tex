\documentclass{beamer}
\usepackage[utf8]{inputenc}
\usepackage{polski}
\usepackage[polish]{babel}
\usetheme{Darmstadt}

\newcommand{\impl}{\rightarrow}
\renewcommand{\implies}{\rightarrow}
\newcommand{\U}{\mathcal{U}}

\title{Homotopiczna teoria typów}

\author{Zeimer}
\date{13 stycznia 2019}

\begin{document}

\frame{\titlepage}

\frame{\tableofcontents}

\section{O co chodzi}

\begin{frame}{Czym jest HoTT?}
\begin{itemize}
	\item Homotopiczna teoria typów (w skrócie HoTT) to połączenie teorii typów i teorii homotopii.
	\item Jest kolejnym stadium ewolucji teorii typów.
	\item Jest syntetyczną teorią homotopii, dającą nam łatwy dostęp do skomplikowanych pojęć topologicznych.
	\item Jest pomysłem na nowe podstawy matematyki, alternatywne wobec teorii zbiorów.
	\item Jest bardzo potężnym funkcyjnym językiem programowania.
	\item Podstawowym źródłem wiedzy jest książka \url{https://homotopytypetheory.org/book/}
\end{itemize}
\end{frame}

\begin{frame}{Teoria typów 1}
\begin{itemize}
	\item Teorię typów w ujęciu HoTTowym można opisać jako system formalny, który za pomocą reguł (osądów) opisuje byty zwane typami. Kluczową innowacją HoTT jest interpretacja typów i wymyślone na jej podstawie aksjomaty rzucające światło na naturę kosmosu.
	\item Reguły dzielą się na ciekawe i nieciekawe.
	\item Nieciekawe to te, które muszą być, żeby wszystko działało, np. do zamieniania kolejności rzeczy w kontekście.
	\item Ciekawe to te, które faktycznie opisują typy. Jest ich pięć rodzajów: reguły formacji, wprowadzania, eliminacji, obliczania i unikalności.
\end{itemize}
\end{frame}

\begin{frame}{Teoria typów 2}
\begin{itemize}
	\item Reguły formacji mówią, skąd się biorą typy.
	\item Reguły wprowadzania mówią, jak zrobić elementy danego typu.
	\item Reguły eliminacji mówią, jak zrobić coś z elementami danego typu.
	\item Reguły obliczania mówią, jak reguły eliminacji mają się do reguł wprowadzania.
	\item Reguły unikalności mówią, jak reguły wprowadzania mają się do reguł eliminacji.
\end{itemize}
\end{frame}

\begin{frame}{Teoria typów 3}

\begin{itemize}
	\item Przykład: reguły dla typu funkcyjnego. Ćwiczenie: nazwij każdą z reguł.
\end{itemize}

	\begin{center}
		$\displaystyle \frac{\Gamma \vdash A : \U \quad \Gamma \vdash B : \U}{\Gamma \vdash A \to B : \U}$
	\end{center}
	\begin{center}
		$\displaystyle \frac{\Gamma, x : A \vdash b : B}{\Gamma \vdash \lambda x:A.b : A \to B}$
	\end{center}
	\begin{center}
		$\displaystyle \frac{\Gamma \vdash f : A \to B \quad \Gamma \vdash x : A}{\Gamma \vdash f\ x : B}$
	\end{center}
	\begin{center}
		$\displaystyle \frac{\Gamma, x : A \vdash b : B \quad \Gamma \vdash a : A}{\Gamma \vdash (\lambda x:A.b)\ a \equiv b[x := a] : B}$
	\end{center}
	\begin{center}
		$\displaystyle \frac{\Gamma \vdash f : A \to B}{\Gamma \vdash \lambda x:A.f\ x \equiv f : A \to B}$
	\end{center}
\end{frame}

\begin{frame}{Teoria homotopii 1}
\begin{itemize}
	\item Co to jest homotopia?
	\item Zgodnie z wikipedią, jeżeli $f$ i $g$ są funkcjami ciągłymi z przestrzeni topologicznej $X$ w przestrzeń topologiczną $Y$, to $H: X \times [0; 1] \to Y$ jest homotopią, gdy jest funkcją ciągłą spełniającą $H(0, x) = f(x) \land H(1, x) = g(x)$.
	\item Jeżeli nieco pogmeramy w symbolach, to możemy to zapisać tak: $H: [0; 1] \to (X \to Y)$ jest homotopią, gdy jest ciągła i spełnia $H(0) = f \land H(1) = g$.
	\item Nie przejmuj się, jeżeli definicja cię nie oświeca. Moim zdaniem władowanie jej do nazwy całej teorii jest głupie.
\end{itemize}
\end{frame}

\begin{frame}{Teoria homotopii 2}
\begin{itemize}
	\item Bardziej podstawowym pojęciem jest ścieżka.
	\item Ścieżka w przestrzeni topologicznej $X$ to funkcja ciągła z $[0; 1]$ w $X$.
	\item Łatwo to sobie wyobrazić: odcinek $[0; 1]$ z pewnością jest ścieżką prowadzącą od $0$ do $1$. Jego obrazem, czyli ścieżką, jest więc pewien ciąły zawijasek, który prowadzi z $f(0)$ do $f(1)$.
	\item Ostatecznie możemy powiedzieć, że homotopia to ścieżka między funkcjami.
	\item Teoria homotopii nie jest jednak teorią ścieżek między funkcjami. Jest to raczej po prostu teoria ścieżek.
\end{itemize}
\end{frame}

\begin{frame}{Teoria homotopii 3}
\begin{itemize}
	\item Po co to wszystko?
	\item Topologia jest całkiem użyteczna. Ostatnio popularna robi się topologiczna analiza danych. Zamiast prymitywnie przypasowywać do danych proste (regresja liniowa), ludzie próbują lepiej opisywać kształt danych. Topologia bada kształty, więc pasuje jak ulał.
	\item Chcemy więc wiedzieć więcej o topologii, np. czy dwie przestrzenie są takie same czy inne. Tutaj wkracza topologia algebraiczna, czyli dziedzina badająca przestrzenie topologiczne za pomocą metod algebraicznych.
\end{itemize}
\end{frame}

\begin{frame}{Teoria homotopii 4}
\begin{itemize}
	\item Pętla w punkcie $x$ to ścieżka, która zaczyna się i kończy w punkcie $x$.
	\item Grupa podstawowa przestrzeni $X$ w punkcie $x$ to grupa, której nośnikiem jest zbiór wszystkich pętli w punkcie $x$. Działaniem grupowym jest sklejanie pętli (najpierw pójdź pierwszą pętlą, a potem drugą). Odwrotność to pójście pętlą w przeciwnym kierunku. Element neutralny to stanie w miejscu.
	\item Grupa podstawowa jest fajna, bo jeżeli przestrzenie są izomorficzne, to ich grupy podstawowe też są. Wobec tego jeżeli grupy podstawowe (w dowolnym punkcie) są różne, to przestrzenie też są różne.
\end{itemize}
\end{frame}

\begin{frame}{Teoria homotopii 5}
\begin{itemize}
	\item Okrąg to taka przestrzeń topologiczna, że... wyobraź sobie, pewnie kiedyś widziałeś okrąg.
	\item Grupa podstawowa okręgu w dowolnym punkcie jest izomorficzna z grupą liczb całkowitych z dodawaniem.
	\item Stanie w miejscu reprezentuje $0$.
	\item $n$ okrążeń zgodnie z ruchem wskazówek zegara reprezentuje liczbę $n$.
	\item $n$ okrążeń przeciwnie do ruchu wskazówek zegara reprezentuje liczbę $-n$.
\end{itemize}
\end{frame}

\begin{frame}{Innowacje HoTT}
\begin{itemize}
	\item Homotopiczna interpretacja teorii typów, mocno wspomagająca wyobraźnię zarówno w rozumowaniu, jak i pozwalająca dogłębnie zrozumieć różne detale teorii typów.
	\item Aksjomat uniwalencji $(A \simeq B) \simeq (A = B)$, który głosi, że rzeczy mające tę samą strukturę są identyczne. Rozwiązuje to odwieczny problem nieformalnego utożsamiania poprzez nadużycie języka.
	\item Wyższe typy induktywne, pozwalające w teorii typów zdefiniować wiele niemożliwych dotychczas obiektów, np. typy ilorazowe, konstruktywnie rozwiązać wiele problemów, które dotychczas wymagały rozumowań klasycznych oraz wyrazić pojęcia czysto logiczne z niemożliwą wcześniej w teorii typów precyzją.
\end{itemize}
\end{frame}

\begin{frame}{Innowacje HoTT w przykładach 1}
\begin{itemize}
	\item Dlaczego w teorii typów (np. w rachunku konstrukcji) nie da się udowodnić, że dla każdego typu $A$ i elementu $x : A$ oraz dowodu równości $p : x = x$ zachodzi $p = refl_x$?
\end{itemize}
\end{frame}

\begin{frame}{Innowacje HoTT w przykładach 2}
\begin{itemize}
	\item Niech typ $\mathbb{N}$ będzie zdefiniowany przez następujące reguły:
	\item Niech $\text{nat} :\equiv 0 \: | \: S\ \text{nat}$
	\item Niech $\text{nat'} :\equiv 0' \: | \: S'\ \text{nat'}$
\end{itemize}
\end{frame}

\begin{frame}{Innowacje HoTT w przykładach 3}
\begin{theorem}
TODO: formatowanie
$\textstyle \prod (A : \mathcal{U}) (B : A \to \mathcal{U}) (R : \Pi x : A, B\ x \to \mathcal{U}),$ \\
	$\textstyle \quad (\prod x : A, \sum y : B\ x, R\ x\ y) \to$ \\
		$\textstyle \qquad \sum f : (\Pi x : A, B x), \Pi x : A, R\ x\ (f\ x)$
\end{theorem}

\begin{itemize}
	\item Jak w teorii typów wyrazić aksjomat wyboru? Zauważmy, że powyższa formulacja nie jest aksjomatem, ale twierdzeniem.
\end{itemize}
\end{frame}

\begin{frame}{Pomysły}
\begin{itemize}
	\item Hierarchia n-typów: unit, funkcje sortujące, pusty, zbiory, grupoidy, okrąg, uniwersum.
	
\end{itemize}
\end{frame}

\begin{frame}{wut}
\begin{itemize}
	\item
\end{itemize}
\end{frame}

\end{document}