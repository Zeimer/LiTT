\documentclass{beamer}
\usepackage[utf8]{inputenc}
\usepackage{polski}
\usepackage[polish]{babel}
\usetheme{Darmstadt}

\newcommand{\impl}{\rightarrow}
\renewcommand{\implies}{\rightarrow}
\newcommand{\U}{\mathcal{U}}

\title{Homotopiczna teoria typów}

\author{Zeimer}
\date{13 stycznia 2019}

\begin{document}

\frame{\titlepage}

\frame{\tableofcontents}

\section{O co chodzi}

\begin{frame}{Odległe związki}
\begin{itemize}
	\item Homotopiczna teoria typów to połączenie teorii homotopii i teorii typów.
	\item Teorii typów nie trzeba nikomu przedstawiać.
	\item Teoria homotopii... TODO.
\end{itemize}
\end{frame}

\begin{frame}{Teoria typów 1}
\begin{itemize}
	\item Teorię typów w ujęciu HoTTowym można opisać jako system formalny, który za pomocą reguł (osądów) opisuje byty zwane typami. Kluczową innowacją HoTT jest interpretacja typów i wymyślone na jej podstawie aksjomaty rzucające światło na naturę kosmosu.
	\item Reguły dzielą się na ciekawe i nieciekawe.
	\item Nieciekawe to te, które muszą być, żeby wszystko działało, np. do zamieniania kolejności rzeczy w kontekście.
	\item Ciekawe - to te, które faktycznie opisują typy. Jest ich pięć rodzajów: reguły formacji, wprowadzania, eliminacji, obliczania i unikalności.
\end{itemize}
\end{frame}

\begin{frame}{Teoria typów 2}
\begin{itemize}
	\item Reguły formacji mówią, skąd się biorą typy.
	\item Reguły wprowadzania mówią, jak zrobić elementy danego typu.
	\item Reguły eliminacji mówią, jak zrobić coś z elementami danego typu.
	\item Reguły obliczania mówią, jak reguły eliminacji mają się do reguł wprowadzania.
	\item Reguły unikalności mówią, jak reguły wprowadzania mają się do reguł eliminacji.
\end{itemize}
\end{frame}

\begin{frame}{Teoria typów 3}
\begin{itemize}
	\item Przykład: reguły dla typu funkcyjnego. Ćwiczenie: nazwij każdą z reguł.
\end{itemize}

	\begin{center}
		$\displaystyle \frac{\Gamma \vdash A : \U \quad \Gamma \vdash B : \U}{\Gamma \vdash A \to B : \U}$
	\end{center}
	\begin{center}
		$\displaystyle \frac{\Gamma, x : A \vdash b : B}{\Gamma \vdash \lambda x:A.b : A \to B}$
	\end{center}
	\begin{center}
		$\displaystyle \frac{\Gamma \vdash f : A \to B \quad \Gamma \vdash x : A}{\Gamma \vdash f\ x : B}$
	\end{center}
	\begin{center}
		$\displaystyle \frac{\Gamma, x : A \vdash b : B \quad \Gamma \vdash a : A}{\Gamma \vdash (\lambda x:A.b)\ a \equiv b[x := a] : B}$
	\end{center}
	\begin{center}
		$\displaystyle \frac{\Gamma \vdash f : A \to B}{\Gamma \vdash \lambda x:A.f\ x \equiv f : A \to B}$
	\end{center}
	
\end{frame}

\begin{frame}{Po co HoTT? 1}
\begin{itemize}
	\item HoTT jest też pomysłem na nowe podstawy matematyki (foundations of mathematics).
	\item Jest teorią typów, więc świetnie nadaje się do formalizacji za pomocą komputerów, czego nie można powiedzieć o teorii zbiorów.
	\item Jest syntetyczną teorią homotopii, co pozwala łatwo formalizować całkiem ezoteryczną matematykę.
	\item Jest konstruktywna, co powinno podobać się konstruktywistom. Co więcej, dzięki wyższym typom induktywnym pozwala na konstruktywne rozwiązanie problemów, które dotąd wymagały podejścia klasycznego.
\end{itemize}
\end{frame}

\begin{frame}{Po co HoTT? 2}
\begin{itemize}
	\item Dopuszcza niektóre klasyczne aksjomaty, co powinno przypaść do gustu klasykom.
	\item Jednocześnie podaje dobre filozoficzne podstawy (i formalne dowody), by niektóre inne klasyczne aksjomaty odrzucić.
	\item Z trzeciej strony, pozwala wyrazić w teorii typów niektóre klasyczne aksjomaty, których dotychczas wyrazić nie było można (np. aksjomat wyboru).
\end{itemize}
\begin{theorem}
$\textstyle
	\prod (A : \mathcal{U}) (B : A \to \mathcal{U}) (R : \Pi x : A, B\ x \to \mathcal{U}),$ \\
		$(\prod x : A, \sum y : B\ x, R\ x\ y) \to$ \\
			$\sum f : (\Pi x : A, B x), \Pi x : A, R\ x\ (f\ x)$
\end{theorem}
\end{frame}


\begin{frame}{Pomysły}
\begin{itemize}
	\item O co chodzi z teorią homotopii i topologią algebraiczną? Grupa podstawowa okręgu.
	\item Hierarchia n-typów: unit, funkcje sortujące, pusty, zbiory, grupoidy, okrąg, uniwersum.
	
\end{itemize}
\end{frame}

\begin{frame}{wut}
\begin{itemize}
	\item
\end{itemize}
\end{frame}

\end{document}