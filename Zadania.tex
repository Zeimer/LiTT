\documentclass[11pt]{article}
\usepackage[margin=0in]{geometry}
\usepackage{amssymb}
\usepackage[T1]{fontenc}
\usepackage[polish]{babel}
\usepackage[utf8]{inputenc}
\usepackage{lmodern}
\usepackage[intlimits]{amsmath}
\usepackage{enumitem}

\usepackage{tikz}
\usetikzlibrary{arrows}

\usepackage{bussproofs}

\selectlanguage{polish}

\renewcommand{\phi}{\varphi}
\newcommand{\LEM}{\text{EM}_\phi}

\title{Rozwiązania zadań z logiki klasycznej i operatorów sterowania}
\author{Zeimer}
\date{15 stycznia 2019}

\begin{document}
    \maketitle
    
	\EnableBpAbbreviations
	
    \par Zad. 1
    \begin{enumerate}[label=(\alph*)]
        \item Chcemy ze specjalnego przypadku prawa Peirce'a $(P\bot)$ oraz prawa eliminacji fałszu $(\bot E)$ wyprowadzić prawo Peirce'a $(P)$. W tym celu w systemie z regułami $(P\bot)$ oraz $(\bot E)$ udowodnimy $\vdash ((\phi \to \psi) \to \phi) \to \phi$.
        
            \begin{prooftree}
                \AXC{}
                \RL{Ass}
                \UIC{$(\phi \to \psi) \to \phi, \phi \to \bot \vdash (\phi \to \psi) \to \phi$}

                \AXC{}
                \RL{Ass}
                \UIC{$(\phi \to \psi) \to \phi, \phi \to \bot, \phi \vdash \phi \to \bot$}

                \AXC{}
                \RL{Ass}
                \UIC{$(\phi \to \psi) \to \phi, \phi \to \bot, \phi \vdash \phi$}

                \BIC{$(\phi \to \psi) \to \phi, \phi \to \bot, \phi \vdash \bot$}

                \RL{$\bot$E}
                \UIC{$(\phi \to \psi) \to \phi, \phi \to \bot, \phi \vdash \psi$}

                \RL{$\to$I}
                \UIC{$(\phi \to \psi) \to \phi, \phi \to \bot \vdash \phi \to \psi$}

                \RL{$\to E$}
                \BIC{$(\phi \to \psi) \to \phi, \phi \to \bot \vdash \phi$}

                \RL{$P\bot$}
                \UIC{$(\phi \to \psi) \to \phi \vdash \phi$}

                \RL{$\to I$}
                \UIC{$\vdash ((\phi \to \psi) \to \phi) \to \phi$}
            \end{prooftree}

        \item Chcemy ze specjalnego przypadku prawa Peirce'a P$\bot$ oraz prawa eliminacji fałszu $\bot$E wyprowadzić prawo eliminacji podwójnej negacji $\neg\neg$E. W tym celu pokażemy w systemie z tymi regułami, że $\vdash ((\phi \to \bot) \to \bot) \to \phi$.
        
            \begin{prooftree}
                \AXC{}
                \RL{Ass}
                \UIC{$(\phi \to \bot) \to \bot, \phi \to \bot \vdash (\phi \to \bot) \to \bot$}

                \AXC{}
                \RL{Ass}
                \UIC{$(\phi \to \bot) \to \bot, \phi \to \bot \vdash \phi \to \bot$}

                \RL{$\to$E}
                \BIC{$(\phi \to \bot) \to \bot, \phi \to \bot \vdash \bot$}

                \RL{$\bot$E}
                \UIC{$(\phi \to \bot) \to \bot, \phi \to \bot \vdash \phi$}

                \RL{P$\bot$}
                \UIC{$(\phi \to \bot) \to \bot \vdash \phi$}

                \RL{$\to$I}
                \UIC{$\vdash ((\phi \to \bot) \to \bot) \to \phi$}
            \end{prooftree}

        \item Chcemy z prawa eliminacji podwójnej negacji $\neg\neg$E wyprowadzić szczególny przypadek prawa Peirce'a P$\bot$ oraz prawo eliminacji fałszu $\bot$E.
        
            \begin{prooftree}
                \AXC{}
                \RL{Ass}
                \UIC{$\bot, \phi \to \bot \vdash \bot$}

                \RL{$\neg\neg$E}
                \UIC{$\bot \vdash \phi$}

                \RL{$\to$I}
                \UIC{$\vdash \bot \to \phi$}
            \end{prooftree}

            \begin{prooftree}
                \AXC{}
                \RL{Ass}
                \UIC{$(\phi \to \bot) \to \phi, \phi \to \bot \vdash \phi \to \bot$}

                \AXC{}
                \RL{Ass}
                \UIC{$(\phi \to \bot) \to \phi, \phi \to \bot \vdash (\phi \to \bot) \to \phi$}

                \AXC{}
                \RL{Ass}
                \UIC{$(\phi \to \bot) \to \phi, \phi \to \bot \vdash \phi \to \bot$}

                \RL{$\to$E}
                \BIC{$(\phi \to \bot) \to \phi, \phi \to \bot \vdash \phi$}

                \RL{$\to$E}
                \BIC{$(\phi \to \bot) \to \phi, \phi \to \bot \vdash \bot$}

                \RL{$\neg\neg$E}
                \UIC{$(\phi \to \bot) \to \phi \vdash \phi$}

                \RL{$\to$I}
                \UIC{$\vdash ((\phi \to \bot) \to \phi) \to \phi$}
            \end{prooftree}
    \end{enumerate}
    
    \newpage

    \par Zad. 2
    \par Udowodnijmy najpierw (w systemie z $\neg\neg$E) prawo wyłączonego środka $\phi \lor \neg \phi$.

    \begin{prooftree}
        \AXC{}
        \RL{Ass}
        \UIC{$(\phi \lor \neg \phi) \to \bot \vdash (\phi \lor \neg \phi) \to \bot$}

        \AXC{}
        \RL{Ass}
        \UIC{$(\phi \lor \neg \phi) \to \bot, \phi \vdash (\phi \lor \neg \phi) \to \bot$}

        \AXC{}
        \RL{Ass}
        \UIC{$(\phi \lor \neg \phi) \to \bot, \phi \vdash \phi$}

        \RL{$\lor I_1$}
        \UIC{$(\phi \lor \neg \phi) \to \bot, \phi \vdash \phi \lor \neg \phi$}

        \RL{$\to$E}
        \BIC{$(\phi \lor \neg \phi) \to \bot, \phi \vdash \bot$}

        \RL{$\to$I}
        \UIC{$(\phi \lor \neg \phi) \to \bot \vdash \neg \phi$}

        \RL{$\lor I_2$}
        \UIC{$(\phi \lor \neg \phi) \to \bot \vdash \phi \lor \neg \phi$}

        \RL{$\to$E}
        \BIC{$(\phi \lor \neg \phi) \to \bot \vdash \bot$}

        \RL{$\neg\neg$E}
        \UIC{$\vdash \phi \lor \neg \phi$}
    \end{prooftree}

    \par Rzut okiem na termy i ich reguły typowania pozwala stwierdzić, że termem odpowiadającym temu dowodowi jest (anotacje typów pozwolę sobie pominąć) $\LEM :\equiv \Delta x. x(\iota_2 (\lambda y. x(\iota_1 y)))$, zaś rzut okiem na reguły redukcji $\to_\Delta$ pozwala stwierdzić, że term ten jest w postaci normalnej (reguły $1$ i $3$ nie pasują, zaś w regule $2$ nie jest spełniona przesłanka dotycząca zmiennej wolnej).
    \par Chcemy teraz znormalizować term $\lambda t. \lambda f. \text{case}(\LEM, x_1.t, x_2.f)$. Normalizujemy więc:

        \begin{center}
            $
            \lambda t. \lambda f. \text{case}(\LEM, x_1.t, x_2.f) \equiv
            \lambda t. \lambda f. \Delta k. (\Delta x. x(\lambda x_1. \lambda x_2. (\lambda y. x(\lambda z_1. \lambda z_2. z_1 y))))(\lambda x_1. kt)(\lambda x_2. kf)
            \to_\Delta
            \lambda t. \lambda f. \Delta k. (\Delta z. (\lambda w. z(w(\lambda x_1. kt)))(\lambda x_1. \lambda x_2. x_2(\lambda y. (\lambda w. z(w(\lambda x_1. kt)))(\lambda z_1. \lambda z_2. y))))(\lambda x_2. kf)
            \to_\beta
            \lambda t. \lambda f. \Delta k. (\Delta z. (\lambda w. z(w(\lambda x_1. kt)))(\lambda x_1. \lambda x_2. x_2(\lambda y. z(\lambda z_2. kt))))(\lambda x_2. kf)
            \to_\beta^*
            \lambda t. \lambda f. \Delta k. (\Delta z. z(\lambda x_2. x_2(\lambda y. z(\lambda z_2. kt))))(\lambda x_2. kf)
            \to_\Delta
            \lambda t. \lambda f. \Delta k. \Delta w. (\lambda a. w(a(\lambda x_2. kf)))(\lambda x_2. x_2(\lambda y. (\lambda a. w(a(\lambda x_2. kf)))(\lambda z_2. kt)))
            \to_\beta^*
            \lambda t. \lambda f. \Delta k. \Delta w. (\lambda a. w(a(\lambda x_2. kf)))(\lambda x_2. x_2(\lambda y. w(kt)))
            \to_\beta
            \lambda t. \lambda f. \Delta k. \Delta w. w((\lambda x_2. x_2(\lambda y. w(kt)))(\lambda x_2. kf))
            \to_\beta
            \lambda t. \lambda f. \Delta k. \Delta w. w((\lambda x_2. kf)(\lambda y. w(kt)))
            \to_\beta
            \lambda t. \lambda f. \Delta k. \Delta w. w(kf)
            \to_\Delta
            \lambda t. \lambda f. \Delta k. kf
            \to_\Delta
            \lambda t. \lambda f. f$
        \end{center}

    \par Jak widać, wychodzi funkcja stale zwracająca drugi argument (czyli fałsz).
\end{document}